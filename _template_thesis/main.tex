% #todo - Template Info!
%----------------------------------------------------------------------------------------
% 	Template: Abschlussarbeiten des Studiengangs Angewandte Informatik an der HTW Berlin
% 	C. Schmidt (Stand: Oktober 2024)
%----------------------------------------------------------------------------------------
%	Pakete und Konfigurationen
%----------------------------------------------------------------------------------------

\documentclass[oneside,bibliography=totocnumbered,BCOR=5mm,numbers=noendperiod]{scrbook}

\usepackage[latin1]{inputenc}
\usepackage{amsmath, amsthm, amssymb}
\usepackage[ngerman]{babel}
%\usepackage[english]{babel} % Language hyphenation and typographical rules
\usepackage{marvosym}
\usepackage{graphicx}
\usepackage{csquotes}
\newtheorem{satz}{Satz}[chapter]
\theoremstyle{definition} 
\newtheorem{definition}[satz]{Definition} 
\theoremstyle{definition} 
\newtheorem{lemma}[satz]{Lemma} 
\theoremstyle{definition} 
\newtheorem{bemerkung}[satz]{Bemerkung}
\theoremstyle{definition} 
\newtheorem{korollar}[satz]{Korollar} 
\theoremstyle{definition}
\newtheorem{beispiel}[satz]{Beispiel} 
\theoremstyle{definition} 
\newtheorem{algorithmus}{Algorithmus} 
\newenvironment{beweis}{\begin{proof}[Beweis]}{\end{proof}}
\usepackage[hyphens]{url}
\usepackage{hyperref}

%----------------------------------------------------------------------------------------
%	BIB.-Datei und Quellenverwaltung
%----------------------------------------------------------------------------------------
\usepackage[
    backend=biber,
    % sorting=none,  % comment back in if you want to sort by order of apperiance / default: sort by authors last name
    style=numeric
]{biblatex}
\DefineBibliographyStrings{ngerman}{
  andothers = {et\addabbrvspace al\adddot}
}
% Add your custom Bibfiles here:
\addbibresource{src/bibliography/custom_bibliography.bib}
%\usepackage{natbib} % use natbib for references 
%----------------------------------------------------------------------------------------
\usepackage{blindtext} % Package to generate dummy text throughout this template 

\usepackage[sc]{mathpazo} % Use the Palatino font
\usepackage[T1]{fontenc} % Use 8-bit encoding that has 256 glyphs
\linespread{1.05} % Line spacing - Palatino needs more space between lines
\usepackage{microtype} % Slightly tweak font spacing for aesthetics

\usepackage[hmarginratio=1:1,top=32mm,columnsep=20pt]{geometry} % Document margins
\usepackage[hang, small,labelfont=bf,up,textfont=it,up]{caption} % Custom captions under/above floats in tables or figures
\usepackage{booktabs} % Horizontal rules in tables
\usepackage{lettrine} % The lettrine is the first enlarged letter at the beginning of the text
\usepackage{enumitem} % Customized lists
\setlist[itemize]{noitemsep} % Make itemize lists more compact

% \usepackage{titlesec} % Allows customization of titles
% \usepackage{titling} % Customizing the title section

%----------------------------------------------------------------------------------------
%	Index
%----------------------------------------------------------------------------------------
\usepackage{makeidx} %Index
\makeindex

%----------------------------------------------------------------------------------------
%	Glossary and Acronyms
%----------------------------------------------------------------------------------------
\usepackage[automake,acronym,toc]{glossaries}
\makeglossaries
% \renewcommand*{\glsnamefont}[1]{\textit{#1}}
% % Example
% \newglossaryentry{dijkstra}{
%     name={Dijkstra},
%     description={Really cool Graph-Algorithm}
% }
% % Example
% \newacronym{reinforcement-learning}{RL}{Reinforcement Learning}

\newglossarystyle{tableacronymstyle}{%
  \setglossarystyle{long} % base style
  \renewenvironment{theglossary}%
    {\begin{longtable}{lp{0.75\linewidth}}}%
    {\end{longtable}}%
  \renewcommand*{\glossaryheader}{%
    \bfseries Abk. & \bfseries Bedeutung \\ \hline\endhead}%
  \renewcommand*{\glossentry}[2]{%
    \glsentryshort{##1} & \glsentrylong{##1} \\}%
}

% \newcommand{\glsnom}[1]{\glsentryuseri{#1}}
% \newcommand{\glsgen}[1]{\glsentryuserii{#1}}
% \newcommand{\glsdat}[1]{\glsentryuseriii{#1}}
% \newcommand{\glsakk}[1]{\glsentryuseriv{#1}}


\usepackage{parskip}
\setlength{\parskip}{1em}
\setlength{\parindent}{0pt}

\usepackage{array}

\usepackage{wrapfig}

\usepackage{subcaption}

%----------------------------------------------------------------------------------------
%	Listings
%----------------------------------------------------------------------------------------
\usepackage{listings}
\usepackage{color}

\definecolor{mygreen}{rgb}{0,0.6,0}
\definecolor{mygray}{rgb}{0.5,0.5,0.5}
\definecolor{mymauve}{rgb}{0.58,0,0.82}

\lstset{ 
  backgroundcolor=\color{white},   % choose the background color; you must add \usepackage{color} or \usepackage{xcolor}; should come as last argument
  basicstyle=\footnotesize,        % the size of the fonts that are used for the code
  breakatwhitespace=false,         % sets if automatic breaks should only happen at whitespace
  breaklines=true,                 % sets automatic line breaking
  captionpos=b,                    % sets the caption-position to bottom
  commentstyle=\color{mygreen},    % comment style
  deletekeywords={...},            % if you want to delete keywords from the given language
  escapeinside={\%*}{*)},          % if you want to add LaTeX within your code
  extendedchars=true,              % lets you use non-ASCII characters; for 8-bits encodings only, does not work with UTF-8
  firstnumber=1,                % start line enumeration with line 1000
  frame=single,	                   % adds a frame around the code
  keepspaces=true,                 % keeps spaces in text, useful for keeping indentation of code (possibly needs columns=flexible)
  keywordstyle=\color{blue},       % keyword style
  language=Python,                 % the language of the code
  mathescape=true,                 % math is executed inside of listings
  morekeywords={*,...},            % if you want to add more keywords to the set
  numbers=left,                    % where to put the line-numbers; possible values are (none, left, right)
  numbersep=5pt,                   % how far the line-numbers are from the code
  numberstyle=\tiny\color{mygray}, % the style that is used for the line-numbers
  rulecolor=\color{black},         % if not set, the frame-color may be changed on line-breaks within not-black text (e.g. comments (green here))
  showspaces=false,                % show spaces everywhere adding particular underscores; it overrides 'showstringspaces'
  showstringspaces=false,          % underline spaces within strings only
  showtabs=false,                  % show tabs within strings adding particular underscores
  stepnumber=1,                    % the step between two line-numbers. If it's 1, each line will be numbered
  stringstyle=\color{mymauve},     % string literal style
  tabsize=2,	                   % sets default tabsize to 2 spaces
  title=\lstname                   % show the filename of files included with \lstinputlisting; also try caption instead of title
}

%----------------------------------------------------------------------------------------
%	Haupttextteil
%----------------------------------------------------------------------------------------

\begin{document}

% Titelseite
% \pagestyle{empty}       % keine Seitennummer
\begin{titlepage}
\begin{center}
\includegraphics{images/HTW_Berlin_Logo_farbig.jpg}
\linebreak[4]
\linebreak[4]
\linebreak[4]
\linebreak[4]
\textit{\large Titel meiner Abschlussarbeit (\#todo)}
\linebreak[4]
\linebreak[4]
\linebreak[4]
Abschlussarbeit 
\linebreak[4]
\linebreak[4]
zur Erlangung des akademischen Grades: 
\linebreak[4]
\linebreak[4]
\textbf{\*Akademischer Grad\* (\#todo)}
% \textbf{Bachelor of Science (B.Sc.)}
% \textbf{Master of Science (M.Sc.)}
\linebreak[4]
\linebreak[4]
an der
\linebreak[4]
\linebreak[4]
Hochschule f\"ur Technik und Wirtschaft (HTW) Berlin
\linebreak[4]
Fachbereich 4: Informatik, Kommunikation und Wirtschaft
\linebreak[4]
Studiengang \textit{Angewandte Informatik}
\linebreak[4]
\linebreak[4]
\linebreak[4]
1. Gutachter\*in: \#todo
\linebreak[4]
2. Gutachter\*in: \#todo
\linebreak[4]
\linebreak[4]
\linebreak[4]
\linebreak[4]
\linebreak[4]
Eingereicht von (\#todo Name) [123456 (\#todo Matr.Nr.)]
\linebreak[4]
\linebreak[4]
\linebreak[4]
\linebreak[4]
\today

\end{center}
\end{titlepage}

\newpage

\thispagestyle{empty}       % keine Seitennummer
% \input{src/vorspann/sperrvermerk.tex}

% \newpage
% \thispagestyle{empty}       % keine Seitennummer
% \input{src/vorspann/danksagung.tex}

\newpage

\thispagestyle{empty}       % keine Seitennummer

\section*{Abstract}
This is the Abstract of my Thesis.

\clearpage
%Seite 1
\pagenumbering{roman}% Seitennummerierung "roemisch"
%\setcounter{page}{1} 

\tableofcontents  



\listoffigures
\listoftables
\lstlistoflistings

\newpage

\pagenumbering{arabic}  % Nummerierung der Seiten in 'arabisch' % neues Kapitel mit Namen "Introduction"
 %Seite 1
 % \setcounter{page}{1}   % setzt Seitenzaehlung auf 1
 
\chapter{Introduction}

Lorem Ipsum

\section{Example}

\section{Sections}

Example usage of an acronym \gls{reinforcement-learning} and a glossary item \gls{dijkstra}.

\subsection{Galore}

This is an example usage of the \verb|\cite{}| command\cite{turhanlar_autonomous_2024}.  % 5%
% \input{src/verwandtearbeiten/verwandtearbeiten.tex} % 10%
% \input{src/grundlagen/grundlagen.tex} % 15%
% \input{src/methodologie/methologie.tex} % 15%
% \input{src/anforderungsanalyse/anforderungsanalyse.tex} % 20%
% \input{src/entwurf/entwurf.tex} % 30%
% \input{src/implementierung/implementierung.tex}
% \input{src/wertung/wertung.tex}
% \input{src/zusammenfassung/zusammenfassung.tex}


%----------------------------------------------------------------------------------------
%	REFERENCE LIST
%----------------------------------------------------------------------------------------


% \bibliographystyle{apalike}
% \bibliographystyle{ksfh_nat} % ein anderer Stil
% \bibliography{science} 

%\begin{thebibliography}{XX}
%\bibitem[Bielecki, Rutkowski(2002)]{bielecki02}T. Bielecki, M. Rutkowski, Credit Risk: Modeling, Valuation and Hedging, Springer (2002).
%\bibitem[Jarrow, Turnbull(1995)]{jarrow95} R.A. Jarrow, S. Turnbull, Pricing derivatives on financial securities subject to credit risk, Journal of Finance, 50:1 (1995) 53--85.
%\bibitem[Marshall, Olkin(1967)]{marshall67} A.W. Marshall, I. Olkin, A multivariate exponential distribution, Journal of the American Statistical Association, 62 (1967), pp. 30--44.
%\bibitem[Sch\"onbucher(2003)]{schoenbucher03} P.J. Sch\"onbucher, Credit Derivatives Pricing Models, Wiley (2003).
%\end{thebibliography}

\printbibliography[
heading=bibintoc,
title={Quellenverzeichnis}
]

\newpage

\printglossary[type=\acronymtype,style=tableacronymstyle,title=Abk\"urzungsverzeichnis, toctitle=Abk\"urzungsverzeichnis]
\printglossary[type=main,title=Glossar, toctitle=Glossar]
\newpage


\begin{appendix}

\pagenumbering{Roman}

% \input{src/appendix/appendix.tex}

\end{appendix}


\end{document}

